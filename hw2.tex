\documentclass{article}
\usepackage[utf8]{inputenc}
\usepackage{amsmath}
\usepackage{amsfonts}
\usepackage{amssymb}
\usepackage{amsthm}
\usepackage{mathrsfs}
\usepackage{mathtools}
\usepackage{fancyvrb}
\usepackage[margin=1in]{geometry}
\usepackage{fancyhdr}
\usepackage[T1]{fontenc}
\usepackage{lmodern}
\usepackage{mathrsfs}
\usepackage{tzplot}
\usepackage{tikz}
\newtheorem{theorem}{Theorem}[section]
\newtheorem{lemma}[theorem]{Lemma}
\usepackage{titlesec}
\usepackage{lmodern}
\usepackage{etoolbox}
\usepackage{csquotes}
\usepackage{xcolor}
\usepackage[shortlabels]{enumitem}
%\usepackage{lipsum}

\makeatletter
\patchcmd{\section}{-3.5ex \@plus -1ex \@minus -.2ex}{-3.5ex \@plus -1ex \@minus -.2ex\setlength{\leftskip}{0cm}}{}{}
\patchcmd{\subsection}{-3.25ex\@plus -1ex \@minus -.2ex}{3.25ex\@plus -1ex \@minus -.2ex\setlength{\leftskip}{0cm}}{}{}
\patchcmd{\subsection}{1.5ex \@plus .2ex}{1.5ex \@plus .2ex\setlength{\leftskip}{2cm}}{}{}
\makeatother
\titleformat{\section}
  {\normalfont\fontsize{12}{15}\bfseries}{\thesection}{1em}{}
\pagestyle{fancy}
\fancyhf{}
\rhead{Aukkawut Ammartayakun}
\lhead{MA 556 Applied Bayesian Statistics: Homework 2}
\cfoot{\thepage}
\title{Homework 2}
\author{Aukkawut Ammartayakun\\MA 556 Applied Bayesian Statistics}
\date{Spring 2024}
\newcommand{\vtt}[1]{%
  \text{\normalfont\ttfamily\detokenize{#1}}%
}
\begin{document}

\maketitle
\noindent
\Large{\textbf{Problem 1}}\normalsize
\\


Explain what you understand by coherence. Tom said that he has diabetes and
hypertension with probability $p$, he has diabetes but not hypertension with probability $q$, and he has diabetes with probability $r$. What must happen to avoid incoherence?

\vspace{\baselineskip}
\noindent
\textbf{Solution}

Coherence means that the situation that is described is following the probability axiom. In this case, let's say the event that Tom has diabetes is $D$ and hypertension $H$. Then, we know that
\begin{align*}
    P(D\cap H) &= p\\
    P(D\cap H^{'}) &= q\\
    P(D) &= r
\end{align*}

As the first two probabilities can be viewed as the partitioning of $D$ by $H$ and $H^{'}$. In this scenario, we know that $(D\cap H) \cap (D\cap H^{'}) = \{\}$ and since those two are partition of $D$, following the Kolmogorov's axiom, $P(D) = P(D\cap H) +  P(D\cap H^{'}) = p+q$.
Thus, $r = p + q$ to be coherent.
\vspace{\baselineskip}

\noindent
\Large{\textbf{Problem 2}}\normalsize
\\


Show that for any $A, B > 0$,
\[A(x-a)^2 + B(x-b)^2 = (A+B)\left(x-\frac{Aa+Bb}{A+B}\right)^2 + \frac{AB}{A+B}\left(a-b\right)^2\]

\vspace{\baselineskip}
\noindent
\textbf{Solution}

Looking at the LHS, we can see that
\begin{align*}
    A(x-a)^2 + B(x-b)^2 &= Ax^2 - 2Aax + Aa^2 + Bx^2 - 2Bbx + Bb^2 \\
    &= (A+B)x^2 - 2(Aa+Bb)x + Aa^2 + Bb^2 \\
    &= (A+B)\left(x^2 - 2\frac{Aa+Bb}{A+B} x + \left(\frac{Aa+Bb}{A+B}\right)^2\right) - \left(\frac{Aa+Bb}{A+B}\right)^2 + Aa^2 + Bb^2\\
    &= (A+B)\left(x - \frac{Aa+Bb}{A+B}\right)^2 - \left(\frac{Aa+Bb}{A+B}\right)^2 + Aa^2 + Bb^2\\
    &=  (A+B)\left(x - \frac{Aa+Bb}{A+B}\right)^2 - \frac{(Aa)^2 + 2ABab + (Bb)^2 + A(A+B)^2a^2+ B(A+B)^2b^2}{(A+B)^2} \\
    &=  (A+B)\left(x-\frac{Aa+Bb}{A+B}\right)^2 + \frac{AB}{A+B}\left(a-b\right)^2
\end{align*}



\newpage
\noindent
\Large{\textbf{Problem 3}}\normalsize
\\

Show that $\frac{P(B)}{P(A)} = \frac{P(B|A)}{P(A|B)}$ where $A$ and $B$ are elements of a Sigma Algebra and $P(\cdot)$ is the associated probability measure. Argue that $\frac{P(B^{'})}{P(A^{'})} = \frac{1-P(B|A^{'})}{1-P(A|B^{'})}$

\vspace{\baselineskip}
\noindent
\textbf{Solution}

We can see that from the definition of $P(B|A)$,
\begin{align*}
    P(B|A) &= \frac{P(A|B) P(B)}{P(A)} \\
    P(A)P(B|A) &= P(B)P(A|B)\\
    \frac{P(B)}{P(A)} &= \frac{P(B|A)}{P(A|B)}
\end{align*}
The conditional probability $P(A|B)$ can be viewed as the reduced sample space from the entire sample space to only $B$ because $B$ already happens. Thus, we can use this intuition to see that $1-P(A|B) = P(A^{'}|B)$. This means
\begin{align*}
    1-P(A|B^{'}) &= P(A^{'}|B^{'})\\
    1-P(A|B^{'}) &= \frac{P(A^{'}\cap B^{'})}{P(B^{'})} 
\end{align*}
Similarly, $1-P(B|A^{'}) = \frac{P(A^{'}\cap B^{'})}{P(A^{'})}$. Thus, the ratios of both two are 
\begin{align*}
    \frac{1-P(B|A^{'})}{1-P(A|B^{'})} &= \frac{\frac{P(A^{'}\cap B^{'})}{P(A^{'})}}{\frac{P(A^{'}\cap B^{'})}{P(B^{'})}}\\
    &= \frac{P(B^{'})}{P(A^{'})}
\end{align*}

\vspace{\baselineskip}
\noindent
\Large{\textbf{Problem 4}}\normalsize
\\

Argue that \[\pi(\theta) \propto \frac{|\theta|}{1+\theta^2},\quad -\infty < \theta < \infty\]
is improper.

\vspace{\baselineskip}
\noindent
\textbf{Solution}

Using $u$-substitution, we can see that

\begin{align*}
    \int \frac{|\theta|}{1+\theta^2}\; d\theta &= \frac{\text{sgn}(\theta)}{2}\ln\left(x^2 + 1\right) + C
\end{align*}

However, looking at the domain of it, we already know that $\lim_{x\rightarrow \infty} \ln(x) = \infty$. Thus, this is a divergence integral in its domain, and it can be argued that it is improper.


\end{document} 
