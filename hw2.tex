\documentclass{article}
\usepackage[utf8]{inputenc}
\usepackage{amsmath}
\usepackage{amsfonts}
\usepackage{amssymb}
\usepackage{amsthm}
\usepackage{mathrsfs}
\usepackage{mathtools}
\usepackage{fancyvrb}
\usepackage[margin=1in]{geometry}
\usepackage{fancyhdr}
\usepackage[T1]{fontenc}
\usepackage{lmodern}
\usepackage{mathrsfs}
\usepackage{tzplot}
\usepackage{tikz}
\newtheorem{theorem}{Theorem}[section]
\newtheorem{lemma}[theorem]{Lemma}
\usepackage{titlesec}
\usepackage{lmodern}
\usepackage{etoolbox}
\usepackage{csquotes}
\usepackage{xcolor}
\usepackage[shortlabels]{enumitem}
%\usepackage{lipsum}
\usepackage{accents}

%stat macro
\newcommand{\ut}[1]{\underaccent{\tilde}{#1}}
\renewcommand{\vec}[1]{\ut{#1}}
\newcommand{\ind}{\stackrel{\text{ind}}{\sim}}
\newcommand{\iid}{\stackrel{\text{iid}}{\sim}}
\newcommand{\cas}{\stackrel{\text{a.s}}{\rightarrow}}
\newcommand{\cp}{\stackrel{p}{\rightarrow}}
\newcommand{\cd}{\stackrel{d}{\rightarrow}}
\newcommand{\nx}{x_1,\dots,x_n}
\newcommand{\ny}{y_1,\dots,y_n}
\newcommand{\Done}[2]{\text{#1}\left({#2}\right)}
\newcommand{\Dtwo}[3]{\text{#1}\left({#2},{#3}\right)}
\newcommand{\cn}{\mathcal{N}}



\makeatletter
\patchcmd{\section}{-3.5ex \@plus -1ex \@minus -.2ex}{-3.5ex \@plus -1ex \@minus -.2ex\setlength{\leftskip}{0cm}}{}{}
\patchcmd{\subsection}{-3.25ex\@plus -1ex \@minus -.2ex}{3.25ex\@plus -1ex \@minus -.2ex\setlength{\leftskip}{0cm}}{}{}
\patchcmd{\subsection}{1.5ex \@plus .2ex}{1.5ex \@plus .2ex\setlength{\leftskip}{2cm}}{}{}
\makeatother
\titleformat{\section}
  {\normalfont\fontsize{12}{15}\bfseries}{\thesection}{1em}{}
\pagestyle{fancy}
\fancyhf{}
\rhead{Aukkawut Ammartayakun}
\lhead{MA 556 Applied Bayesian Statistics: Homework 2}
\cfoot{\thepage}
\title{Homework 2}
\author{Aukkawut Ammartayakun\\MA 556 Applied Bayesian Statistics}
\date{Spring 2024}
\newcommand{\vtt}[1]{%
  \text{\normalfont\ttfamily\detokenize{#1}}%
}
\begin{document}

\maketitle
\noindent
\Large{\textbf{Problem 1}}\normalsize
\\


It is believed that $35\%$ of Americans belong to the democratic party. In an opinion poll it was found that $20\%$ of those belonging to the democratic party voted yes and $60\%$ of those belonging to other parties voted yes. Suppose someone selected at random voted yes, can you tell whether she/he is a democratic?

\vspace{\baselineskip}
\noindent
\textbf{Solution}

Let $D$ be an event that a random person is democratic, and $Y = 1$ means that person vote yes for the poll and $Y = 0$ means otherwise. Then,
\[P(D) = 0.35, \quad P(Y = 1|D) = 0.20, \quad P(Y = 1|D^{'}) = 0.60 \]
Then,
\begin{align*}
    P(D|Y=1) &= \frac{P(Y=1|D)P(D)}{P(Y=1|D)P(D) + P(Y=1|D^{'}) P(D^{'})}\\
    &= \frac{(0.20)(0.35)}{(0.20)(0.35) + (0.60)(0.65)}\\
    &= 0.1522
\end{align*}
Thus, we can say that with the probability of 0.1522, that person is democratic.
\vspace{\baselineskip}

\noindent
\Large{\textbf{Problem 2}}\normalsize
\\


I tossed a coin 25 times independently and I got 25 heads. What does this
information tell you about the next toss?  [Hint: You can assume a priori that the probability of heads is a uniform random variable.]


\vspace{\baselineskip}
\noindent
\textbf{Solution}

Let $H$ be our prior and assume that $H\sim \text{Unif}(0,1)$. Then, let $E$ be our evidence (number of heads. in this case, 25 heads). This essentially follows the model
\begin{align*}
    E|H,p &\sim \text{Binom}(n,p)\\
    p &\sim \text{Unif}(0,1)
\end{align*}
Here, this is Binomial-Beta conjugate (as $\text{Unif}(0,1)$ is equivalent to $\text{Beta}(1,1)$), and the result should be Beta distribution. To show that, assume that $p\sim \text{Beta}(\alpha,\beta)$, then
\begin{align*}
    H|E = x &\propto {n\choose x} p^{x}(1-p)^{n-x} \frac{p^{\alpha-1}(1-p)^{\beta - 1}}{B(\alpha,\beta)}\\
    &\propto p^{x+\alpha -1} (1-p)^{n-x+\beta - 1}\\
    &\sim \text{Beta}(x+\alpha, n-x+\beta)
\end{align*}
In this case, $x = 25$, $n = 25$, $\alpha = \beta = 1$. Then, $H|E = 25 \sim \text{Beta}(26,1)$. That means, the probability that this will be head given the evidence is $\mathbb{E}[H|E] = \frac{26}{27} = 0.963$.

\newpage
\noindent
\Large{\textbf{Problem 3}}\normalsize
\\

Explain the importance of the likelihood principle. Why can Bayesians and
non-Bayesians disagree about the likelihood principle? What should your stand be? Is the mode (point of maximum) of a likelihood function a random variable?

\vspace{\baselineskip}
\noindent
\textbf{Solution}

We can see that 

\vspace{\baselineskip}
\noindent
\Large{\textbf{Problem 4}}\normalsize
\\

The joint density of $x_1,\dots x_n$ is
\[f(\vec{x}) = \frac{\beta^{\alpha}\Gamma(n\bar{x} + \alpha)}{\{\prod_{i=1}^n x!\}\Gamma(\alpha)(\beta + n)^{n\bar{x} + \alpha}}, \; x = 0,1,\dots\]

where $\alpha$ and $\beta$ are fixed known numbers.
\begin{enumerate}
    \item Are $\nx$ independent? Are $\nx$ identical? Are $\nx$ exchangeable?
    \item Find a random variable $y$ such that given $y$, $\nx$ are independent and identically distributed.
\end{enumerate}

[Hint: Consider Poisson and Gamma distributions.]

\vspace{\baselineskip}
\noindent
\textbf{Solution}

\begin{enumerate}
    \item $\nx$ are exchangeable because, in this density, each $x_i$ operates under the operation with permutation-invariant (product for the denominator and addition in the mean). Thus, it is exchangeable.

    To show that this is independent or identical, we need to show each distribution of this joint density is independent or identical
    \item sub 2
\end{enumerate}

\vspace{\baselineskip}
\noindent
\Large{\textbf{Problem 5}}\normalsize
\\

Let 
\begin{align*}
   \ny, y_{n+1}, \dots y_N | \mu & \ind\mathcal{N}(\mu,\sigma^2)\\
    \mu &\sim \mathcal{N}(0,\delta^2)
\end{align*}
Find the posterior density of $\bar{Y} = \frac{1}{N}\sum_{i=1}^N y_i$ if $\ny$ are observed (data).


\vspace{\baselineskip}
\noindent
\textbf{Solution}

x

\vspace{\baselineskip}
\noindent
\Large{\textbf{Problem 6}}\normalsize
\\

Let $\ny | \theta \ind \Done{Bernoulli}{\theta}, \theta \sim \Dtwo{Unif}{0}{1}$. Let $\phi = \ln\left(\frac{\theta}{1-\theta}\right)$. Find a
simulation consistent estimator of $\mathbb{E}[\phi|\vec{y}]$. Suppose $n = 100$ and $\sum_{i=1}^n y_i =25$, find the estimate and give its standard error.


\vspace{\baselineskip}
\noindent
\textbf{Solution}

x
\end{document} 
